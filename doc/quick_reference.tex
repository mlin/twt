\documentclass{article}
\usepackage[top=0.5in,bottom=0.5in,left=0.5in,right=0.5in]{geometry}

\begin{document}
\pagestyle{empty}

\setlength{\columnseprule}{1pt}
\twocolumn
\section*{{\tt ocaml+twt 0.90} quick reference}
\noindent This sheet tries to concisely demonstrate most syntax forms recognized by the {\tt ocaml+twt} preprocessor. If you need more details, check the examples included with the distribution. All structural whitespace in the following examples is significant: if a line is indented here, it must be indented in your source.

\subsection*{Applications}

\begin{verbatim}
List.iter
 Printf.printf "%d\n"
 lst

List.map
 function Some x -> x
 List.filter
  function
   | Some x -> true
   | None -> false
  lst

(if b then (+) else (-))
 x
 y
\end{verbatim}

\subsection*{Sequences}

Nothing special:

\begin{verbatim}
statement-1
statement-2
...
\end{verbatim}

\subsection*{let}

\noindent Let looks more like it would in a procedural language:

\begin{verbatim}
let x = 1
sequence

let rec f x =
 sequence
and g x =
 sequence
and h y =
 sequence
sequence

let x = match y with
 | pat1 -> csq1
 | pat2 -> csq2
sequence
\end{verbatim}

\begin{verbatim}
let x = 1
let y = 2
sequence
\end{verbatim}

\noindent You can't put a let and its consequent on one line (e.g. {\tt let x = y in f x}).

\subsection*{if-then-else}

\begin{verbatim}
if condition then expression

if condition then
 sequence

if condition then expression else expression

if condition then
 sequence
else
 sequence

if condition then
 sequence
else if condition then
 sequence
else
 sequence 
\end{verbatim}

\subsection*{fun}

Nothing special, but you don't need parentheses if the {\tt fun} is on its own line:
\begin{verbatim}
fun x y -> expression
fun x y ->
 sequence
\end{verbatim}

\subsection*{Pattern matching}

All patterns occurring on their own line must be indented and have pipes:

\begin{verbatim}
function Some x -> true | None -> false

match expression with
 | pattern -> expression
 | pattern ->
    sequence
 | pattern -> expression
\end{verbatim}

\noindent A {\tt match} or {\tt function} may appear on the same line as a {\tt let}:

\begin{verbatim}
let x = function
 | pattern -> expression
 ...
sequence
\end{verbatim}

\subsection*{Exception handling}

Nothing special, given the above forms for pattern matching:

\begin{verbatim}
try expression with Exception -> expression
 
try
 sequence
with
 | Exn1 -> expression
 | Exn2 ->
    sequence

\end{verbatim}

\subsection*{Records, lists, and arrays}

\noindent The preprocesor ignores anything within curly braces or square brackets, including newlines. Thus, indentation within these operators doesn't matter, you still have to use {\tt ;} to separate items, and complicated expressions must be parenthesized.

\begin{verbatim}
type point = { x : int; y : int }
type point = {
     x : int;
   y : int
}

if condition then
 [ 1; 2 ]
else
 [ 2;
   1 ]

let x = [elem1; elem2]
sequence

let x = [elem1;
         elem2;
         ...   ]
sequence

let x =
 { field1 = value1;
   field2 = value2;
   ...             }
sequence
\end{verbatim}

\subsection*{Loops}

Don't use {\tt done}:

\begin{verbatim}
for i = 1 to 10 do expression

for i = 1 to 10 do
 sequence

while condition do expression

while condition do
 sequence
\end{verbatim}

\subsection*{Modules}

Don't use {\tt end}:

\begin{verbatim}
module A = struct
 type point = { x : int;
                y : int }

 let origin = { x = 0; y = 0 }

 let reflect_x { x = a; y = b } =
   { x = 0 - a;
     y = b }
\end{verbatim}

\noindent The local module syntax is supported, but the {\tt struct} must start on its own line. For functors, {\tt module Name = functor ... ->} must appear as one line. See the {\tt modules.ml} example for details.

\noindent Module signatures are the same, except without {\tt end}:

\begin{verbatim}
module A : sig
 type point = { x : int;
                y : int }
 val origin : point
 val reflect_x : point -> point
\end{verbatim}

\subsection*{Objects}

\noindent Don't use {\tt end}. Method and initializer bodies are any other sequence. See the {\tt objects.ml} example for details.

\begin{verbatim}
class shape =
object
 method virtual area : unit -> float
 
class circle =
object (self)
 inherit shape
 val r = 1.0
 method area () =
  3.14159 *. r *. r
\end{verbatim}

\subsection*{Union types}

\noindent The syntax rules for pattern matching apply:

\begin{verbatim}
type shape =
 | Square
 | Circle
 | Triangle
\end{verbatim}

\subsection*{Combining expressions}

\noindent Because {\tt ocaml+twt} is a line-oriented preprocessor, the following general rule applies when combining expressions on the same line:

\vspace{0.5cm}

\noindent \textbf{If an expression spans multiple lines, it must start on its own line.}

\vspace{0.5cm}

\noindent For example, the following will \textbf{not} work:

\begin{verbatim}
let lst2 = List.map 
            string_of_float
            lst1
sequence
\end{verbatim}

\noindent The multi-line application in this example must start on a new line, rather than on the same line as its containing expression. (Alternatively, you could just make it one parenthesized line.) There are a few exceptions to this rule: loops, local module structures, and immediate objects must always start on their own line (even if they are only one line), {\tt match} and {\tt function} may appear on the same line as a {\tt let} (even if their patterns are on individual lines), and records, lists, and arrays may span multiple lines as previously described.

\end{document}